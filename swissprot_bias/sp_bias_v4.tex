% Template for PLoS
% Version 1.0 January 2009
%
% To compile to pdf, run:
% latex plos.template
% bibtex plos.template
% latex plos.template
% latex plos.template
% dvipdf plos.template

\documentclass[10pt]{article}

% amsmath package, useful for mathematical formulas
\usepackage{amsmath}
% amssymb package, useful for mathematical symbols
\usepackage{amssymb}

% graphicx package, useful for including eps and pdf graphics
% include graphics with the command \includegraphics
\usepackage{graphicx}

% cite package, to clean up citations in the main text. Do not remove.
\usepackage{cite}

\usepackage{color} 

% Use doublespacing - comment out for single spacing
%\usepackage{setspace} 
%\doublespacing


% Text layout
\topmargin 0.0cm
\oddsidemargin 0.5cm
\evensidemargin 0.5cm
\textwidth 16cm 
\textheight 21cm

% Bold the 'Figure #' in the caption and separate it with a period
% Captions will be left justified
\usepackage[labelfont=bf,labelsep=period,justification=raggedright]{caption}

% Use the PLoS provided bibtex style
\bibliographystyle{plos2009}

% Remove brackets from numbering in List of References
\makeatletter
\renewcommand{\@biblabel}[1]{\quad#1.}
\makeatother


% Leave date blank
\date{}

\pagestyle{myheadings}
%% ** EDIT HERE **


%% ** EDIT HERE **
%% PLEASE INCLUDE ALL MACROS BELOW

%% END MACROS SECTION

\begin{document}

% Title must be 150 characters or less
\begin{flushleft}
{\Large
\textbf{Title}
}
% Insert Author names, affiliations and corresponding author email.
\\
\author{Alexandra Schnoes$^1$%
       \email{Alexandra Schnoes - alexandra.schnoes@ucsf.edu}%
      \and
         Alexander Thorman$^2$%
         \email{Alexander Thorman - thormanaaw@muohio.edu}
       and 
        Iddo Friedberg\correspondingauthor$^{2,3}$%
        \email{Iddo Friedberg\correspondingauthor - i.friedberg@muohio.edu}
      }
      

%Author1$^{1}$, 
%Author2$^{2}$, 
%Author3$^{3,\ast}$
\address{%
    \iid(1)Department of Bioengineering and Therapeutic Sciences, University of California San Francisco, San Francisco, CA, USA\\
    \iid(2)Department of Microbiology, Miami University, Oxford, OH USA\\
    \iid(3)Department of Computer Science and Software Engineering , Miami University, Oxford, OH USA
}%
\\
\bf{1} Author1 Dept/Program/Center, Institution Name, City, State, Country
\\
\bf{2} Author2 Dept/Program/Center, Institution Name, City, State, Country
\\
\bf{3} Author3 Dept/Program/Center, Institution Name, City, State, Country
\\
$\ast$ E-mail: Corresponding author@institute.edu
\end{flushleft}

% Please keep the abstract between 250 and 300 words
\section*{Abstract}

        \paragraph*{Background:} Computational protein function
prediction programs rely upon well-annotated databases for testing and training
their algorithms. These databases, in turn, rely upon the work of curators
to capture experimental findings from scientific literature and apply them to protein
sequence data. However, with the increasing use of high-throughput experimental
assays,  a small number of experimental papers dominate
the functional protein annotations collected in databases. 
Here we investigate just how prevalent is the ``few papers --
many proteins'' phenomenon. We hypothesize that the dominance of high-throughput experiments in
proteins annotation biases our view of the corpus of functions enabled by proteins.
      
        \paragraph*{Results:} We examine the annotation of
UniProtKB by the Gene Ontology Annotation project (GOA), and show that
the distribution of proteins per paper is a log-odd, with 0.06\% of papers
dominating 20\% of the annotations. Since each of the dominant papers
describes the use of an assay that can find only one function or a small
group of functions, this leads to substantial biases, in several aspects, in what we know
about the function of many proteins.

        \paragraph*{Conclusions:} Given the experimental techniques
available, protein function annotation bias due to high-throughput experiments is unavoidable. Knowing
that these biases exist and understanding their characteristics and extent is important for
database curators, developers of function annotation programs, and
anyone who uses protein function annotation data to plan experiments.

\end{abstract}
% Please keep the Author Summary between 150 and 200 words
% Use first person. PLoS ONE authors please skip this step. 
% Author Summary not valid for PLoS ONE submissions.   
\section*{Author Summary}

\section*{Introduction}

Functional annotation of proteins is a primary challenge in molecular biology
today\cite{Friedberg2006Automated,Erdin2011180,Rentzsch2009210}. The continuing revolution in sequencing technology
means that the conversation has shifted from realizing the \$1000 genome to the one-hour genome \cite{PMID
Stahl2012Toward}. The ability to rapidly and cheaply sequence genomes is creating a flood of sequence data, which require
extensive analysis and characterization before they can be useful.  A large proportion of this work involves
assigning biological function to these newly determined gene sequences, a process that is both complex and costly
\cite{Sboner2011Real}.  Furthermore,  the ability to accurately assign function through computational
means is challenging and open problem \cite{Schnoes2009Annotation}. To aid current annotation procedures
and improve computational function prediction algorithms, sources of high-quality, experimentally derived
functional data are necessary.  Currently, one of the few repositories of such data is the UniProt-GOA database
\cite{Dimmer2012UniProtGO}, which contains both computationally derived and literature derived functional information. The
literature derived information is extracted by human curators who capture functional data from publications, assign
the data to its appropriate place in the Gene Ontology hierarchy \cite{Ashburner2000Gene} and label them with appropriate
functional evidence codes. The UniProt-GOA database is one of only a small number of databases that explicitly
connects functional data, publication references and evidence codes to specific, experimentally studied sequences.
In addition, annotations captured in UniProt-GOA directly impact the annotations in the UniProt/Swiss-Prot
database, widely considered the largest gold standard set of functional annotation \cite{swissprot ref, Schnoes
2009} available. 

It is important, therefore, to understand which trends and biases are
encapsulated by the UniProt-GOA database, as those impact well-used sister
databases and function prediction algorithm development and training. One
concern surrounding the capture of functional data from papers is the
propensity for high-throughput experimental work to become a large fraction of
the data in UniProt-GOA, this having few experiments dominate the protein function landscape.

In this work we analyzed the relative contribution of papers to the experimental annotations
in UniProt-GOA.  We found some striking biases, stemming from the finding that a small
fraction of papers that describe high-throughput experiments, disproportionately contribute
to the pool of experimental annotations of model organisms. Consequently, we show that: 1)
annotations coming from high-throughput experiments are mostly less informative than those
provided by low-throughput experiments;  2) annotations from high throughput experiments
bias the annotations towards a limited number of functions, and, 3) many high-throughput
experiments overlap in the proteins they annotate, and in the annotations assigned. Taken
together, our findings offer a comprehensive picture of how the current protein function
landscape is generated. Furthermore, due to the biases inherent in the current system of
sequence annotations, this study serves as a caution to the producers and consumers of these
data.


% Results and Discussion can be combined.
\section*{Results}

\subsection*{Articles and Proteins} With the advent of high-throughput experiments it has become
possible to conduct large-scale studies of protein functions.  Consequently, some studies reveal
very specific functional aspects of a large amount of proteins as a result of the particular type
of assay or assays used. To understand the impact of large-scale studies on the corpus of
experimentally annotated proteins, we looked at the UniprotKB Gene Ontology annotation files, or
UniProt-GOA. UniProt-GOA proteins are individually annotated by one or more GO terms.
using a procedure described in \cite{Uniprot-GOA}. Briefly, this procedure consists
of six steps which include sequence curation, sequence motif analyses, literature-based curation,
reciprocal BLAST\cite{Altschul1997Gapped} searches, attribution of all resources leading to the included
findings, and quality assurance. If the annotation source is a research article, the attribution
includes its PubMed ID. For each GO term associated with a protein, there is also an
\textit{evidence code} which is used to explain how the association between the protein and the
GO term was made.  Experimental evidence codes include such terms as: Inferred by Direct Assay
(IDA) which indicates that ``a direct assay was carried out to determine the function, process,
or component indicated by the GO term'' or Inferred from Physical Interaction (IPI) which
``Covers physical interactions between the gene product of interest and another molecule.''
(Taken from the GO site, geneontology.org).  Computational evidence codes include terms such
as \textit{Inferred from Sequence or Structural Similarity} (ISS) and \textit{Inferred from
Sequence Orthology} (ISO).  However, these are still assigned by a curator. There are also
non-computational and non-experimental evidence codes, the most prevalent being \textit{Inferred
from Electronic Annotation} (IEA) which is ``used for annotations that depend directly on
computation or automated transfer of annotations from a database''. IEA evidence means that the
annotation was not made or checked by a person.  Different degrees of reliability are associated
with the evidence codes, with experimental codes generally considered to be of higher reliability
than non-experimental codes. However, the increase in the number of high-throughput experiments
used to determine protein functions may introduce biases into protein annotations, due to the
inherent capabilities and limitations of high-throughput assays.  

To test the hypothesis that such biases exist, and to assess their extent if they
do, we compiled the details of all experimentally-annotated
proteins in UniProtKB. This included all proteins whose GO annotations have
the GO experimental evidence codes EXP, IDA, IPI, IMP, IGI, IEP. We first examined
the distribution of articles by the number of proteins they annotate. The results
are shown in Figure~\ref{fig:papers-prots}. 

As can be seen in Figure\ref{fig:papers-prots}, the distribution of the number of proteins
annotated per paper follows a power-law distribution, $f(x)=a\dot x^k$. 
Using the goodness-of-fit based
method we found a significant fit to $a=7; k=2.59$. 
We therefore conclude that there is indeed a substantial bias in
experimental annotations, in which there are few papers that annotate a large
number of proteins.

To better understand the consequences of such a distribution, we divided the annotating
articles into four cohorts, based on the number of proteins each article annotates.
\textit{Single-throughput} papers are those papers that annotate only a single protein;
\textit{low throughput} papers annotate 2-9 proteins; \textit{moderate throughput} papers
annotate 10-99 proteins and \textit{high throughput} papers annotate over 99 proteins. The
results are shown in Table~\ref{tab:cohorts}. High throughput papers are responsible for
21\% of the annotations in Uniprot-GOA, even though they comprise 0.07\% of the papers. 66\%
of the papers are single-throughput and low throughput, however those annotate 53\% of the
proteins in Uniprot-GOA. So while moderate throughput and high-throughput experiments
account for slightly under half of the annotations in Uniprot-GOA, the number of experiments
in those cohorts is much smaller than low throughput experiments and single-throughput
experiments. 

What typifies high-throughput papers? Also, how may the log-odds distribution bias what we
understand of the protein function universe? To answer these questions, we examined
different aspects of the annotations in the four paper cohorts. Also, we examined in higher
detail the top 50 annotating papers. (Overall, 62 papers in our study annotated more than
100 proteins). 

An initial characterization of the top 50 high-throughput papers is shown in
Table\ref{table:top-papers}.  As can be seen, almost all of the papers are specific to a
single species (typically a model organism) and assay that is used to annotate the proteins
in that organism.  Since a single assay was used, then typically only one ontology (MFO, BPO
or CCO) was used for annotation. Therefore, the biases we are seeing are almost always
per-species and per-ontology. For some species this means that a single functional aspect
(MFO, BPO or CCO) of a species will be dominated by a single experiment.

\subsection*{Term frequency bias}
To see how much a single species-- and method-- specific large-scale assay affects
the entire annotation of a species, we examined the relative contribution of the
top-50 papers to the entire corpus of experimentally annotated protein in each
species.  All the species we examined were model organisms, as all the top
annotation-contributing papers dealt with model organisms.  The results are
summarized in Figure\ref{fig:rel-contrib}. % "Annotation Trends" in Alex's poster

Terms that were found to be frequent were: 

% The next paragraph should be should be moved to Methods. We'll just write that we removed the
% propagation bias. This is not a finding, it's a technical clean-up.
Another type of annotation bias may result due to GO-structure derived redundancy: the annotation of
a single term using a given GO-term and one or more of its parents. However, this type of
annotation may not necessarily be wholly redundant. For example, a protein may localize to
the nucleolus and the nucleus itself. In the Cellular Compartment ontology, ``nucleus'' is
a parent term for ``nucleolus''. In the case of this hypothetical protein, annotating with
``nucleolus'' and ``nucleus'' is not an error. However, if the protein was found to
localize to the nucleolus only, then annotating with both terms is a redundancy. We termed
such a redundancy \textit{propagation bias}. We examined all proteins for possible propagation
bias. If a protein was annotated with a GO term and one or more parent terms, the parent
terms were removed. We found possible propagation biases in 17 of the 50 papers. The results
are summarized in Table\ref{tab:prop-bias}. 

\subsection*\{Ontology bias}

The term frequency bias appears to be a direct result of the ontology bias described here.
The proteins annotated by single-protein papers, low-throughput papers, and moderate
throughput papers (less than 100) have similar ratios of the fraction of proteins annotated.
Twenty-two to twenty-six percent of assigned terms are in the Molecular Function Ontology,
and 51-57\% are in the Biological Process Ontology and the remaining 17-25\% are in the
Cellular Component ontology. This ratio changes dramatically with high-throughput papers
(over 99 terms per paper). Now only 5\% of assigned terms are in the Molecular Function
Ontology, 38\% in the Biological Process Ontology and 57\% in the Cellular Compartment
Ontology, ostensibly due to a lack of high-throughput assays that can be used for
generating annotations using the Molecular Function Ontology. 

\subsection*{Confirmation Bias}

% Text for dream-catcher plots
Another type of annotation bias is that of protein re-annotation. How many of the top-50 papers
actually re-annotate the same set of proteins? On the one hand, independent experimental
confirmation does have it merits. On the other hand, repeated similar annotations of the same
proteins in the same organisms using similar assays may be simply indicate a confirmation bias.
To investigate the extent of repetitive annotations in different papers, we clustered all the
proteins annotated by the top-50 papers using CD-HIT\cite{CD_HIT} at 100\% sequence identity. We
then examined the number of clusters containing 100\% identical sequences per model species. The
product of the number of proteins divided by the number of clusters is the redundancy
percentage. For example, if each of the top-50 papers annotating the proteins in a given species
annotated the same protein set, the redundancy percentage would be 100\%. The results of
confirmation bias analysis are shown in Figure~\ref{fig:dreamcatcher1} and in
Table~\ref{tab:dreamcatcher1}. As can be seen, the highest percent redundancy is among the ZZZ
papers annotating {\em C. elegans}. 

We have determined, therefore, that there is a some degree of repetition between experiments in
the proteins they annotate, with some overlap being quite high. However, there is still the
need to determine the extent of the repetition of the annotation. We therefore analyzed the
100\% sequence identity clusters for overlap in annotation.  To do so, we counted the number
of identical GO-terms per ontology within each cluster, and divided that by the sum of
GO-terms shared between all papers in the cluster. The result is a number between 0 and 1.
Zero means no GO-terms are shared, while one means all GO-terms are shared. 

The results are shown in Figure~\ref{fig:dreamcatcher2} and in
Table~\ref{tab:dreamcatcher2}. In \textit{S.  cerevisiae}, four papers contribute to the
Cellular Component ontology, by annotating 635 proteins which are common between two or more
of these papers. Among those proteins,  79.6\% of the terms produced are identical.

\subsection*{Quantifying annotation information}

A common assumption holds that while high-throughput experiments do annotate more protein
functions than low-throughput experiments, the former also tend to be more shallow in the
predictions they provide. The information provided, for example, by a large-scale protein
binding assay will only tell us if two proteins are binding, but will not reveal whether
that binding is specific, will not provide an exact $K_{bind}$, will not say under what
conditions binding takes place, or whether there is any enzymatic reaction or
signal-transduction involved. Having on hand data from experiments with different
``thorughputness'' levels,  we set out to investigate whether there is a difference in the
information provided by high-throughput experiments vs. low-throughput ones. To answer this
question, we first have to quantify the information given by GO terms. One way to do so, is
to use the depth of the term in the ontology: the term ``enzyme activity'' would be less
informative than ``dehalogenase'' and the latter will be less informative than ``haloalkane
dehalogenase''.  We therefore counted edges from the ontology root term to the GO-term to
determine term information. The larger the number of edges, the more specific -- and
therefore informative -- the annotation. In cases where several paths lead from the root to
the examined GO-term, we used the minimal path. We did so for all the annotating papers
split into groups by the number of proteins each paper annotates. 


Edge counting provides a measure of term-specificity. It is, however, imperfect. The reason is
that different areas of the GO DAG have different connectivities, and terms may have different
depths unrelated to the intuitive specificity of a term. For example ``high-affinity
Tryptophan transporter'', (GO:0005300) is 14 terms deep, while ``anticoagulant'', (GO:0008435)
is only three terms deep.  For this reason, information content, the logarithm of the inverse
of the GO term frequency in the corpus is generally accepted as a measure of GO-term
information content\cite{lord-semsim}. Therefore, to account for the possible bias created by
the GO-DAG structure, we also used the log frequency of the terms in the experimentally
annotated proteins in Uniprot-GOA. However, it should be noted that the log-frequency measure
is also imperfect because, as we see throughout this study, a GO-term's frequency may be
heavily influenced  by the top annotating papers, injecting a circularity problem into the use
of this metric in the first place.
Since no single metric for measuring the information
conveyed by a GO term is wholly satisfactory, we used both in this study.

The results of both analyses are shown in Figure~\ref{fig:go-depth} and the accompanying
Table~\ref{tab:go-depth}. In general, the results from the depth-based analysis and the
log-frequency based analysis are in agreement, when compared across groupings based on the
number of proteins annotated by the papers. For the Molecular Function ontology, the
distribution of edge counts and log-frequency scores decreases as the number of annotated
proteins per-paper increases.  For the Biological Process ontology, the decrease is
significant. However the contributer to the decrease are the high-throughput papers
while there is little change in the first three paper cohorts.
Finally, there is no significant trend of GO-depth decrease in the Cellular Component
Ontology. However, using the information content metric, there is also a significant decrease
in information content in the high-throughput paper cohort.

\subsection*{Annotation consistency}
Another interesting question was how consistent were the annotations between different experiments?


\subsection*{Evidence and Assertion}

There are two complementary ways by which we come to knowledge about a protein's function. The approximately 20 GO
evidence codes, discussed above, encapsulate the type results by which the function was inferred, but they do not
capture all the necessary information. For example, ``Inferred by Direct Assay (IDA)'' informs that experimental
evidence was used, but does not say which type. The current Evidence Code Ontology used in GO does not offer that
information.  The newer ECO ontology provides \textit{assertion terms} in addition to evidence terms, in which the
nature of the assay is given. For example, an enzyme-linked immunoabsorbent assay (ELISA) provides quantitative
protein data \textit{in vitro} while an immunogold assay may provide the same information, and cellular
localization information \textit{in vivo}. It is therefore important to know both the assertion and the evidence to
understand what sort of information may be gleaned from the assay. At the moment, Uniprot is not using assertion
codes. However, to understand which types of assertions are made in the top-50 high throughput papers, we performed
a manual curation of these papers. The results are shown in Figure~{fig:assertion} and in Table~{tab:assertion}.



\subsection*{Annotation quality}

One reflection on annotation coverage is the number of GO terms assigned to any
given protein. Ostensibly, the larger the number of GO terms that are assigned to a
protein, the more comprehensive its annotation, provided that these terms are
non-redundant, i.e. not direct parents of each other. 

The median number of annotations per protein was 1.09, which means that most of the top-50
papers do not provide more than a single GO-term per protein. Also, the mean number of
annotations per protein was 1.59. The difference between the mean and median reflects that
most papers have an annotations-per-protein ratio which is close to 1, whereas few have a
higher annotation ratio. As shown in Figure~\ref{fig:annotation-ratio}, that is essentially
the case. Seven papers had an annotations per proteins ratio which is above 2. Those
were paper numbers 1, 4, 10, 23, 34 and 43 on the list. We decided to examine these papers
to see whether these studies did indeed provide better coverage, and what distinguished them
from the other high throughput studies. To follow are brief summaries of the
methodologies used in these papers.

\textbf{Toward a confocal subcellular atlas of the human proteome.}

In this microscopy-based study, the authors used specific staining techniques and confocal
microscopy to describe the subcellular localization of 4937 proteins, which were each assigned
to one or more of ten different subcellular compartments. Since proteins may be assigned to more
than a single compartment, there is a mean of 2.23 GO annotations per protein. The most frequent
terms were ``nucleus'' and ``cytosol'' with 2211 and 2208 terms, respectively. The next
most frequent terms was``cytoplasm'', ``endoplasmatic reticulum'' and ``golgi apparatus'' with
400-500 terms each.  ''\cite{PMID:11121744}. This paper is the top annotating one that we found,
both in GO-terms per protein, and in total number of proteins annotated. 

\textbf{RNAi-based studies of \textit{C. elegans}}

In \cite{PMID:17417969} the authors fed various bacterial clones expressing dsRNA to \textit{C.
elegans} mutants that are hypersensitive to dsRNA interference. As  a result, they discovered
novel loss-of-function phenotypes for 393 \textit{C. elegans} genes. In total they reported 5918
annotations for 1791 proteins.  The functions reported here were for the Biological Process
Ontology. Figure 2B 

The three other studies were also RNAi-based using \textit{C. elegans}. 


\section*{Discussion}

We have identified several annotation biases in UniProt-GOA. These biases stem from the uneven number of
annotations produced by different types of experiments. It is clear that results from high-throughput experiments
contribute heavily to the function annotation landscape. At the same time, these experiments produce less
information per protein than moderate--, low-- and single--\~throughput experiments as evidenced by the type of
terms produced in the Molecular Function and Biological Process ontologies. Furthermore, the number of total GO
terms used in the high-throughput experiments is much lower than that used in low and medium throughput
experiments. Therefore, while high throughput experiments provide a higher coverage of protein
function space per experiment, it is the low throughput experiments that provide information richness. 

We have also identified several types of biases that contributed by high throughput experiments. First, there is
the enrichment of low-information content GO-terms. 

Taken together, these annotation biases affect our perception of protein function space. Several steps can be taken
to remedy this situation. Proteins whose annotation is derived solely from high-throughput experiments can be flagged
as such in the database. In a typical use-case scenario, a researcher will BLAST their query protein to determine
its function by sequence similarity. If a target protein is tagged as annotated by a high throughput
assay, this can provide more information to the user with which to act. Ideally, all GO annotated proteins should
also be annotated with assertion codes in addition to the evidence codes and GO term-codes; but given the large
volume of data in UniprotKB is it hard to expect such massive reannotation undertaken.





% You may title this section "Methods" or "Models". 
% "Models" is not a valid title for PLoS ONE authors. However, PLoS ONE
% authors may use "Analysis" 
\section*{Materials and Methods}

% Do NOT remove this, even if you are not including acknowledgments
\section*{Acknowledgments}


%\section*{References}
% The bibtex filename
\bibliography{template}

\section*{Figure Legends}
%\begin{figure}[!ht]
%\begin{center}
%%\includegraphics[width=4in]{figure_name.2.eps}
%\end{center}
%\caption{
%{\bf Bold the first sentence.}  Rest of figure 2  caption.  Caption 
%should be left justified, as specified by the options to the caption 
%package.
%}
%\label{Figure_label}
%\end{figure}


\section*{Tables}
%\begin{table}[!ht]
%\caption{
%\bf{Table title}}
%\begin{tabular}{|c|c|c|}
%table information
%\end{tabular}
%\begin{flushleft}Table caption
%\end{flushleft}
%\label{tab:label}
% \end{table}

\end{document}

